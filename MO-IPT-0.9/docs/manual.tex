\documentclass[a4paper]{article}
\usepackage[affil-it]{authblk}
\usepackage{graphicx}
%\usepackage{epsfig}
\usepackage{amsmath}
\usepackage{setspace}
\topmargin=-1.0in
%\rightmargin=-1.0in
%\leftmargin=0.4in
%\setlength{\textheight}{9.9in}
%\setstretch{0.9}
%\setlength{\textwidth}{6.3in}
%\usepackage[margin=0.8in]{geometry}


\begin{document}

\title{\bf User-guide for MO-IPT package.}
\author{Wasim Raja Mondal}
\author{Nagamalleswararao Dasari} 
\author{N.\ S.\ Vidhyadhiraja}
\affil{Theoretical Sciences Unit, \\ Jawaharlal Nehru Centre for Advanced Scientific
Research, \\ Jakkur, Bangalore 560064, India.}
\maketitle

\section{Brief description}
MO-IPT stands for multi-orbital iterated perturbation theory. The present code
implements the MO-IPT solver as described in Dasari et al [paper reference] within
the dynamical mean field theory (DMFT) framework and may be used to obtain single-
-particle spectra and self-energies for strongly correlated model Hamiltonians as
well as real materials with multiple orbital degrees of freedom. The code is implemented
for zero as well as finite temperatures. The impurity solver uses the second-order
self-energy in an ansatz motivated by the continued fraction expansion of the self-
-energy. The ansatz has certain free parameters that are chosen to satisfy high
frequency and the atomic limits. Since the ansatz reproduces low frequency Fermi liquid
behaviour and the band limit by construction, the MO-IPT is expected to be a reasonable
interpolating approximation. Naturally, the MO-IPT cannot be expected to be accurate
close to phase transitions, etc, where exact methods such as QMC and NRG would be
far more accurate. For extensive benchmarking of the method with continuous time
Monte Carlo and other methods, please see Dasari et al [paper reference]. The main
limitation of the code is that the Hund's coupling is presently considered only as
a density-density interaction. The main merits of the MO-IPT is that it is fast,
numerically inexpensive, can deal with many orbitals, and provides real frequency
results at zero and finite temperature. The main motivation of our implementation
is to carry out first principles calculations of strongly correlated materials, so
the integration with band structure results (from e.g WIEN2K) is also implemented
in this set of codes.


\section{Quick build and execution}

\begin{itemize}
\item{cd MO-IPT/src; make clean}
\item{make}
\item{cd ..; chmod 755 run.sh; ./run.sh} 
\end{itemize}

This is an interactive script.  There are five sample data sets. For the 1$^{\rm st}$
and 2$^{\rm nd}$ data sets, the number of recommended MPI processes, $np$, is one, while for
the rest, some speedup is achieved by $np > 1$. 

\section{Directory Structure}
\text{The package has the following directory structure:}
\begin{description}
\item [README]
\item [COPYRIGHT.txt]  
\item [docs/] Documentation
\begin{itemize}
\item manual.pdf  $ $  $ $  // Detailed instructions and guide
\item flowchart.pdf $ $ $ $ // Flowchart for MO-IPT
\end{itemize}
\item [data/] Example calculations. Each sample set has an {\tt input/} and an 
{\tt output/}
subdirectory.
\begin{itemize}
\item sample\_1/ and sample\_2/ $  $ // Examples for model Hamiltonian calculations with toy density of states
\item sample\_3/ and sample\_4/ $ $ // Examples for real material calculations
\item sample\_5/ // Example for a model Hamiltonian calculation on a 
2-D square lattice
\end{itemize} 
\item [src/] Source files
\begin{itemize}
\item Makefile
\item main.f90  $ $ $ $ // Main program for MO-IPT calculation, DMFT iterations, self-energy calculation etc.
\item makegrid.f90 $ $ $  $ // Module for creating real frequency (energy) grids.
\item ksum.f90     $ $ $ $ // Module for momentum summation and constructing local Green's functions.
\item funct.f $ $ $ $ // Collection of auxiliary modules like Fermi function, Hilbert
transform, interpolation, root finder etc. 
\item Glob\_cons   $ $ $ $ // Global constants.
\end{itemize}
\end{description}

\section{Input/Output description}

\textbf {The input files are described below:}

\begin{itemize}
\item {par.dat}
\text{This is the most important input file for MO-IPT calculations. The structure of this file is given below}\\
\\
\text{$b_1$ $ $ $\epsilon_1$ $ $  $\epsilon_2$ $ $ $b_2$ $ $ $db_2$ $ $ $\epsilon_3$} \;
\;\;\;\; Grid parameters \\
\\
\text{$b_3$ $ $ $db_3$ $ $ $\epsilon_4$ $ $ $\epsilon_5$} \;\;\;\;\;\;\;\;\;\;\;\;
\;\;\;\;\;Grid parameters\\
\\
\text{$U$ $ $ $J_H$ $ $ $dfac$ $ $ $correc$ $ $ $correc2$ $ $ $t$ $ $ $\alpha$ $ $ $Jflag$}\\
\\
\text{$phsym$ $ $ $LDA\_DMFT$ $ $ $init$ $ $ $idos$ $ $ $uniform\_grid$}\\
\\
\text{$temp$ $ $ $frac$ $ $ $eta$} \\
\\
\text{$Norbs$ $ $ $Nkp$ $ $ $NQPTS$ $ $ $Ndim$}
\end{itemize}

\subsection{Input parameter description}
 \begin{description}
 
  \item The first two lines define an inhomogeneous frequency or energy grid, and are used in
  makegrid.f90. 
  \item [$U$:] represents local Coloumb repulsion (Hubbard - U).
  \item [$J_H$:] stands for Ising type Hund's coupling.
  \item [$dfac$:] This is mixing parameter between old and new solutions, and is used
  for smoother convergence.
  \item [$correc$, $correc2$ :] These parameters are mixing fractions for two particle correlation (2PC) and three particle correlation (3PC) functions. So if $correc=0.3$,
  then only 0.3 of the 2PC function enters the calculations. It is advised to keep
  $correc$ small for initial calculations and increase them gradually using converged data for lower $correc$ parameters. 
  \item [$t$:] Bare hopping integral for model calculations and in general is the unit of energy.
  \item [$Jflag$:] This flag decides the format of $U-J$ matrix.
 
 \item[$phsym$:] This flag can be used to enforce particle-hole symmetry. If $phsym=0$,
 then the chemical potential is tuned to get the fixed total occupancy, else if
 $phsym=1$, the occupancy of each orbital is fixed at $0.5$ and the $\mu$ is set
 to zero. 

\item[$LDA\_DMFT$:] If $idos > 3$, then two options are available using the
$LDA\_DMFT$ flag. If this flag is set to 1, then DMFT calculations will be carried out for a DFT-calculated $H_0(k)$. So an input file names HK\_data will be expected.
The number of Q-points used for the DFT calculation also need to be specified in the
$NQPTS$ variable (see below). For other values of $LDA\_DMFT$ (and $idos \ge 4$),
the model calculations with real density of states (2-D square or 3-D cubic) 
are carried out. The dimensionality $Ndim$ needs to be specified in this case.



  \item [$init$:] init=0 means - start a fresh calculation and no previous data is available. init=1 means supply previously computed data as input.

  \item [$idos$:] Flag for choice of non-interacting density of states.
  
                 idos=1 : Hypercubic lattice, Gaussian Density of states. 
                  
                  idos=2 : Bethe lattice or Cayley tree, semicircular density of states. 
                  
                  idos=3 : General (user supplied) density of states. 
                  
                  idos=4 : Real material calculations using LDA Tight - Binding input OR
                           model calculations on a 2D square/3D cubic lattice. 

\item [$uniform\_grid$:] If this flag is set to 1, then a homogeneous $\omega$-grid
is created, else a logarithmic+homogeneous grid is generated.

  \item [$temp$:] Temperature in units of $t$ for model calculations or in eV when doing
  real-material calculations. 
  \item [$frac$:] Set default value as 0.
  \item [$eta$:] Defines appropriate contour in the complex plane. It should be positive and typical value is of the order of $10^{-4}$.   
\item [$Norbs$:] is the total number of orbitals (including spin degeneracy). If $N$ is the
 number of orbitals you are interested, then $Norbs$ will take value $2N$. The factor $2$ is
  because each orbital is considered to have two spin indices ($\uparrow$ or $\downarrow$).
\item [$Nkp$:] is the total number of k-points from 0 to $\pi$, and is made use of to generate
a k-point mesh for real lattice calculations (with idos=4, LDA\_DMFT=0, Ndim=2 or 3). The default
value is 0.

\item [$NQPTS$:] The total number of K-points in the $H_0(k)$ produced by the DFT.  For example,
in the sample\_3 and sample\_4, the DFT was carried out using a $25\times25\times$ K-point mesh, hence
the $H_0(k)$ contains 15625 entries, and thus NQPTS=15625. This parameter is used only when
doing real-material calculations (idos=4, LDA\_DMFT=1). The default value is 0.

\item [$Ndim$:] The dimensionality of the lattice. This parameter is used in conjunction with
 $Nkp$ when model calculations need to be performed on realistic 2D square or 3D cubic lattices.
 For these $Ndim=2$ or $3$ respectively must be specified. See also the description for
 $LDA\_DMFT$ above.

 \end{description}



\begin{itemize}
\item initials.dat: This is another input file that contains initial occupancies and
chemical potentials. The structure of the file is given below:\\
\text{$n_{1\uparrow}$}\\
\text{$n_{1\downarrow}$}\\
\text{$n_{2\uparrow}$}\\
\text{$n_{2\downarrow}$}\\
\text{.}\\
\text{.}\\
\text{.}\\
\text{.}\\
\text{.}\\
\text{.}\\
\text{.}\\
\text{$n_{N\uparrow}$}\\
\text{$n_{N\downarrow}$}\\
\text{$\mu_{guess}$, $\mu_{0\,guess}$, $Fermi Level (DFT)$}\\
\text {$ntotinput$}
\end{itemize}

\begin{description}
\item [$occupancies$]: $n_{r\uparrow}$ is the occupancy of the r$^{\rm th}$  orbital for up spin and $n_{r\downarrow}$ is the occupancy of the r$^{\rm th}$ orbital for down spin.
\item [$\mu_{guess}$]: User-provided guess value of $\mu$ so that total occupancy constraint is satisfied. The code finds appropriate $\mu$ given a fixed total occupancy, $n_{tot}$.
\item [$\mu_{0\, guess}$]: User-provided guess pseudo-chemical potential for satisfying 
Luttinger's Theorem (LT). The code does not modify this automatically, and at present a manual
tuning of $\mu_0$ is required to satisfy the LT.
\item [$Fermi Level (DFT)$]: For real material calculations, it is the Fermi energy calculated by DFT.
\item [$ntotinput$] Total occupancy of all the orbitals considered in the calculation.
This is a fixed number for a given calculation.
\end{description}



\begin{itemize}
\item{HK\_data}\\

This input file represents a non-interacting Hamiltonian matrix in the orbitals
basis for several momentum values. Mathematically, this is denoted as $H_0({\mathbf{k}})$. This input is needed when $idos=4$ and $LDA_DMFT=1$.
which is the combination appropriate for first principles based real material calculations and
is derived from DFT calculations followed by downfolding. The structure of this input file is given below:
\begin{equation}
H_{0}(k) =
\begin{pmatrix}
P_n(k_1) \\
P_n(k_2) \\
\vdots \\
\vdots \\
P_n(k_{NQPTS})
\end{pmatrix}
\end{equation}

where each of the $P_n(k)$ are one-dimensional column vectors given by 
\begin{equation}
P_{n}(k) =
 \begin{pmatrix}
H_{1\uparrow,1\uparrow}(k) \\ H_{1\uparrow,1\downarrow}(k)\\ H_{1\uparrow,2\uparrow}(k)\\ H_{1\uparrow,2\downarrow}(k) \\ \cdots \\ H_{1\uparrow,n\uparrow} \\ H_{1\uparrow,n\downarrow} \\
H_{1\downarrow,1\uparrow}(k) \\ H_{1\downarrow,1\downarrow}(k)\\ H_{1\downarrow,2\uparrow}(k) \\ H_{1\downarrow,2\downarrow}(k) \\ \cdots \\ H_{1\downarrow,n\uparrow} \\ H_{1\downarrow,n\downarrow} \\
  \vdots  \\ \vdots  \\
H_{n\uparrow,1\uparrow}(k) \\ H_{n\uparrow,1\downarrow}(k)\\    H_{n\uparrow,2\uparrow}(k)\\ H_{n\uparrow,2\downarrow}(k)\\ \cdots \\ H_{n\uparrow,n\uparrow} \\ H_{n\uparrow,n\downarrow} \\
H_{n\downarrow,1\uparrow}(k) \\ H_{n\downarrow,1\downarrow}(k) \\ H_{n\downarrow,2\uparrow}(k) \\   H_{n\downarrow,2\downarrow}(k) \\ \cdots \\ H_{n\downarrow,n\uparrow} \\ H_{n\downarrow,n\downarrow} 
\end{pmatrix}
\end{equation}
\end{itemize}



\textbf{The following are the main output files:}

\begin{itemize}
\item{Green's functions}

The computed Green's functions are written in Gf1.dat for the 1$^{\rm st}$ orbital,
spin $\uparrow$,
Gf2.dat for the 1$^{\rm st}$ orbital, spin $\downarrow$, Gf3.dat for the 2$^{\rm nd}$ orbital, spin $\uparrow$, and so on.

In each file there are three columns. The first column is frequency, the second is
the positive definite spectral function and the third column is the real part of the Green's function.

\item{Self energy}

The calculated self-energies are written in sig1up.dat file for the 1$^{\rm st}$ orbital,
 up spin and sig1down.dat file for the 1$^{\rm st}$ orbital, down spin. And
similarly for the 2$^{\rm nd}$ orbital, sig2up.dat and sig2down.dat files contain
self-energies for up and down spins respectively. In each file, frequency is in the 1st column, imaginary part of the self-energy is in the second column and real part of the self energy is 
in the third column.

\item{quasiwei.dat}

Calculated quasi-particle weights are written in this file.

\item interaction matrix.dat

The interaction matrix is written in this file.


\item Mastubara\_self.dat

The self-energies computed on the real frequency axis are Fourier transformed
to Matsubara frequencies and then written to this file. The first column is
the $\omega_n=(2n+1)\pi T$, where $T$ is the temperature, and the rest of the columns
are imaginary parts of the up-spin self-energies of the orbitals, i.e
Im($\Sigma_{1\uparrow}(i\omega_n))$, Im($\Sigma_{2\uparrow}(i\omega_n))$
and so on. Currently $n\in [-300:300]$.


\item Mastubara\_green.dat 

The Green's functions computed on the real frequency axis are Fourier transformed
to Matsubara frequencies and then written to this file. The format is similar to the self-energy file above, except that the self-energy is replaced by the corresponding Green's function.

\end{itemize}


\section{System information}
The codes have been tested on the following system configurations:

\begin{itemize}
\item{Ubuntu on Intel Xeon Hex-core E5645 (2.4GHz) with 12GB RAM}
\begin{itemize}
\item{\tt uname -a}
\begin{description}
\item Linux 3.2.0-23-generic $36$ -Ubuntu SMP Tue Apr 10 20:39:51 UTC 2012 x86\_64 x86\_64 x86\_64 GNU/Linux
\end{description} 
\item{\tt lsb\_release -a}
\begin{description}
\item DISTRIB ID=Ubuntu
\item DISTRIB RELEASE = $12.04$
\item DISTRIB CODENAME=precise
\item DISTRIB DESCRIPTION= Ubuntu 12.04.4 LTS 
\end{description}

\item{\tt mpif90 -v}

mpif90 for MPICH version 3.0.3

ifort version 13.0.0
\end{itemize}

\item{Mageia on Intel Core i7-4765T (2GHz) with 8GB RAM }
\begin{itemize}
\item{\tt uname -a}
\begin{description}
\item Linux 3.10.28-desktop-1.mga3 \#1 SMP Sat Feb 1 16:15:10 UTC 2014 x86\_64 x86\_64 x86\_64 GNU/Linux
\end{description} 
\item{\tt lsb\_release -a}
\begin{description}
\item Distributor ID: Mageia
\item Description:    Mageia 3
\item Release:        3
\item Codename:       thornicroft
\end{description}

\item{\tt mpif90 -v}

mpif90 for MPICH2 version 1.2.1

ifort version 14.0.1


\end{itemize}

\end{itemize}

\section{Examples}

These examples are located in {\tt MO-IPT/data/}.
\begin{itemize}


\item{\bf sample\_1:} Model calculation of the doped two-orbital degenerate Hubbard Model for density-density type interaction
at $T=0$. The $U=1.5$, $J_H=0$, and the total filling, $n_{tot}=1.6$. The bare
density of states is a Gaussian (Hypercubic lattice). The relevant files are in
{\tt data/sample\_1}.

\item{\bf sample\_2:} Model calculation of the half-filled non-degenerate two-orbital Hubbard model for density-density type interactions. The system is half-filled 
($n_{tot}=2$) but not p-h symmetric. The $U=1.5$ and $J_H=0$. The relevant files are in
{\tt data/sample\_2}.

\item{\bf sample\_3:} Real material calculation for SrVO$_3$ at $T=0$. The DFT input, $H_0(k)$, has been obtained through WIEN2K and Wannier90 through standard procedures.
The $U=1.2$ and $J_H=0.1$. The relevant files are in
{\tt data/sample\_3}.

\item{\bf sample\_4:} Real material calculation for SrVO$_3$ at $T=0.1 eV$. The DFT input, $H_0(k)$, has been obtained through WIEN2K and Wannier90 through standard procedures.
The relevant files are in
{\tt data/sample\_4}.

\item{\bf sample\_5:} Model calculation of the half-filled, p-h symmetric 
two-orbital Hubbard model at $U=1.5$, and $J_H=0$ on a 2-D square lattice
with 120 k-points along each direction.
The relevant files are in
{\tt data/sample\_5}.
\end{itemize}

\section{Flowchart}

The full flowchart of the LDA+DMFT method is displayed in {\tt docs/flowchart.pdf}. A description of the symbols may be found in 
Dasari et al [].

\end{document}
